\documentclass[journal]{IEEEtran}
% *** CITATION PACKAGES ***
%
\usepackage{cite}
\usepackage{hyperref}
% cite.sty was written by Donald Arseneau
% V1.6 and later of IEEEtran pre-defines the format of the cite.sty package
% \cite{} output to follow that of the IEEE. Loading the cite package will
% result in citation numbers being automatically sorted and properly
% "compressed/ranged". e.g., [1], [9], [2], [7], [5], [6] without using
% cite.sty will become [1], [2], [5]--[7], [9] using cite.sty. cite.sty's
% \cite will automatically add leading space, if needed. Use cite.sty's
% noadjust option (cite.sty V3.8 and later) if you want to turn this off
% such as if a citation ever needs to be enclosed in parenthesis.
% cite.sty is already installed on most LaTeX systems. Be sure and use
% version 5.0 (2009-03-20) and later if using hyperref.sty.
% The latest version can be obtained at:
% http://www.ctan.org/pkg/cite
% The documentation is contained in the cite.sty file itself.






% *** GRAPHICS RELATED PACKAGES ***
%
\ifCLASSINFOpdf
  % \usepackage[pdftex]{graphicx}
  % declare the path(s) where your graphic files are
  % \graphicspath{{../pdf/}{../jpeg/}}
  % and their extensions so you won't have to specify these with
  % every instance of \includegraphics
  % \DeclareGraphicsExtensions{.pdf,.jpeg,.png}
\else
  % or other class option (dvipsone, dvipdf, if not using dvips). graphicx
  % will default to the driver specified in the system graphics.cfg if no
  % driver is specified.
  % \usepackage[dvips]{graphicx}
  % declare the path(s) where your graphic files are
  % \graphicspath{{../eps/}}
  % and their extensions so you won't have to specify these with
  % every instance of \includegraphics
  % \DeclareGraphicsExtensions{.eps}
\fi
% graphicx was written by David Carlisle and Sebastian Rahtz. It is
% required if you want graphics, photos, etc. graphicx.sty is already
% installed on most LaTeX systems. The latest version and documentation
% can be obtained at: 
% http://www.ctan.org/pkg/graphicx
% Another good source of documentation is "Using Imported Graphics in
% LaTeX2e" by Keith Reckdahl which can be found at:
% http://www.ctan.org/pkg/epslatex
%
% latex, and pdflatex in dvi mode, support graphics in encapsulated
% postscript (.eps) format. pdflatex in pdf mode supports graphics
% in .pdf, .jpeg, .png and .mps (metapost) formats. Users should ensure
% that all non-photo figures use a vector format (.eps, .pdf, .mps) and
% not a bitmapped formats (.jpeg, .png). The IEEE frowns on bitmapped formats
% which can result in "jaggedy"/blurry rendering of lines and letters as
% well as large increases in file sizes.
%
% You can find documentation about the pdfTeX application at:
% http://www.tug.org/applications/pdftex





% *** MATH PACKAGES ***
%
%\usepackage{amsmath}
% A popular package from the American Mathematical Society that provides
% many useful and powerful commands for dealing with mathematics.
%
% Note that the amsmath package sets \interdisplaylinepenalty to 10000
% thus preventing page breaks from occurring within multiline equations. Use:
%\interdisplaylinepenalty=2500
% after loading amsmath to restore such page breaks as IEEEtran.cls normally
% does. amsmath.sty is already installed on most LaTeX systems. The latest
% version and documentation can be obtained at:
% http://www.ctan.org/pkg/amsmath





% *** SPECIALIZED LIST PACKAGES ***
%
%\usepackage{algorithmic}
% algorithmic.sty was written by Peter Williams and Rogerio Brito.
% This package provides an algorithmic environment fo describing algorithms.
% You can use the algorithmic environment in-text or within a figure
% environment to provide for a floating algorithm. Do NOT use the algorithm
% floating environment provided by algorithm.sty (by the same authors) or
% algorithm2e.sty (by Christophe Fiorio) as the IEEE does not use dedicated
% algorithm float types and packages that provide these will not provide
% correct IEEE style captions. The latest version and documentation of
% algorithmic.sty can be obtained at:
% http://www.ctan.org/pkg/algorithms
% Also of interest may be the (relatively newer and more customizable)
% algorithmicx.sty package by Szasz Janos:
% http://www.ctan.org/pkg/algorithmicx




% *** ALIGNMENT PACKAGES ***
%
%\usepackage{array}
% Frank Mittelbach's and David Carlisle's array.sty patches and improves
% the standard LaTeX2e array and tabular environments to provide better
% appearance and additional user controls. As the default LaTeX2e table
% generation code is lacking to the point of almost being broken with
% respect to the quality of the end results, all users are strongly
% advised to use an enhanced (at the very least that provided by array.sty)
% set of table tools. array.sty is already installed on most systems. The
% latest version and documentation can be obtained at:
% http://www.ctan.org/pkg/array


% IEEEtran contains the IEEEeqnarray family of commands that can be used to
% generate multiline equations as well as matrices, tables, etc., of high
% quality.




% *** SUBFIGURE PACKAGES ***
%\ifCLASSOPTIONcompsoc
%  \usepackage[caption=false,font=normalsize,labelfont=sf,textfont=sf]{subfig}
%\else
%  \usepackage[caption=false,font=footnotesize]{subfig}
%\fi
% subfig.sty, written by Steven Douglas Cochran, is the modern replacement
% for subfigure.sty, the latter of which is no longer maintained and is
% incompatible with some LaTeX packages including fixltx2e. However,
% subfig.sty requires and automatically loads Axel Sommerfeldt's caption.sty
% which will override IEEEtran.cls' handling of captions and this will result
% in non-IEEE style figure/table captions. To prevent this problem, be sure
% and invoke subfig.sty's "caption=false" package option (available since
% subfig.sty version 1.3, 2005/06/28) as this is will preserve IEEEtran.cls
% handling of captions.
% Note that the Computer Society format requires a larger sans serif font
% than the serif footnote size font used in traditional IEEE formatting
% and thus the need to invoke different subfig.sty package options depending
% on whether compsoc mode has been enabled.
%
% The latest version and documentation of subfig.sty can be obtained at:
% http://www.ctan.org/pkg/subfig




% *** FLOAT PACKAGES ***
%
%\usepackage{fixltx2e}
% fixltx2e, the successor to the earlier fix2col.sty, was written by
% Frank Mittelbach and David Carlisle. This package corrects a few problems
% in the LaTeX2e kernel, the most notable of which is that in current
% LaTeX2e releases, the ordering of single and double column floats is not
% guaranteed to be preserved. Thus, an unpatched LaTeX2e can allow a
% single column figure to be placed prior to an earlier double column
% figure.
% Be aware that LaTeX2e kernels dated 2015 and later have fixltx2e.sty's
% corrections already built into the system in which case a warning will
% be issued if an attempt is made to load fixltx2e.sty as it is no longer
% needed.
% The latest version and documentation can be found at:
% http://www.ctan.org/pkg/fixltx2e


%\usepackage{stfloats}
% stfloats.sty was written by Sigitas Tolusis. This package gives LaTeX2e
% the ability to do double column floats at the bottom of the page as well
% as the top. (e.g., "\begin{figure*}[!b]" is not normally possible in
% LaTeX2e). It also provides a command:
%\fnbelowfloat
% to enable the placement of footnotes below bottom floats (the standard
% LaTeX2e kernel puts them above bottom floats). This is an invasive package
% which rewrites many portions of the LaTeX2e float routines. It may not work
% with other packages that modify the LaTeX2e float routines. The latest
% version and documentation can be obtained at:
% http://www.ctan.org/pkg/stfloats
% Do not use the stfloats baselinefloat ability as the IEEE does not allow
% \baselineskip to stretch. Authors submitting work to the IEEE should note
% that the IEEE rarely uses double column equations and that authors should try
% to avoid such use. Do not be tempted to use the cuted.sty or midfloat.sty
% packages (also by Sigitas Tolusis) as the IEEE does not format its papers in
% such ways.
% Do not attempt to use stfloats with fixltx2e as they are incompatible.
% Instead, use Morten Hogholm'a dblfloatfix which combines the features
% of both fixltx2e and stfloats:
%
% \usepackage{dblfloatfix}
% The latest version can be found at:
% http://www.ctan.org/pkg/dblfloatfix




%\ifCLASSOPTIONcaptionsoff
%  \usepackage[nomarkers]{endfloat}
% \let\MYoriglatexcaption\caption
% \renewcommand{\caption}[2][\relax]{\MYoriglatexcaption[#2]{#2}}
%\fi
% endfloat.sty was written by James Darrell McCauley, Jeff Goldberg and 
% Axel Sommerfeldt. This package may be useful when used in conjunction with 
% IEEEtran.cls'  captionsoff option. Some IEEE journals/societies require that
% submissions have lists of figures/tables at the end of the paper and that
% figures/tables without any captions are placed on a page by themselves at
% the end of the document. If needed, the draftcls IEEEtran class option or
% \CLASSINPUTbaselinestretch interface can be used to increase the line
% spacing as well. Be sure and use the nomarkers option of endfloat to
% prevent endfloat from "marking" where the figures would have been placed
% in the text. The two hack lines of code above are a slight modification of
% that suggested by in the endfloat docs (section 8.4.1) to ensure that
% the full captions always appear in the list of figures/tables - even if
% the user used the short optional argument of \caption[]{}.
% IEEE papers do not typically make use of \caption[]'s optional argument,
% so this should not be an issue. A similar trick can be used to disable
% captions of packages such as subfig.sty that lack options to turn off
% the subcaptions:
% For subfig.sty:
% \let\MYorigsubfloat\subfloat
% \renewcommand{\subfloat}[2][\relax]{\MYorigsubfloat[]{#2}}
% However, the above trick will not work if both optional arguments of
% the \subfloat command are used. Furthermore, there needs to be a
% description of each subfigure *somewhere* and endfloat does not add
% subfigure captions to its list of figures. Thus, the best approach is to
% avoid the use of subfigure captions (many IEEE journals avoid them anyway)
% and instead reference/explain all the subfigures within the main caption.
% The latest version of endfloat.sty and its documentation can obtained at:
% http://www.ctan.org/pkg/endfloat
%
% The IEEEtran \ifCLASSOPTIONcaptionsoff conditional can also be used
% later in the document, say, to conditionally put the References on a 
% page by themselves.




% *** PDF, URL AND HYPERLINK PACKAGES ***
%
%\usepackage{url}
% url.sty was written by Donald Arseneau. It provides better support for
% handling and breaking URLs. url.sty is already installed on most LaTeX
% systems. The latest version and documentation can be obtained at:
% http://www.ctan.org/pkg/url
% Basically, \url{my_url_here}.




% *** Do not adjust lengths that control margins, column widths, etc. ***
% *** Do not use packages that alter fonts (such as pslatex).         ***
% There should be no need to do such things with IEEEtran.cls V1.6 and later.
% (Unless specifically asked to do so by the journal or conference you plan
% to submit to, of course. )


% correct bad hyphenation here
%\hyphenation{op-tical net-works semi-conduc-tor}


\begin{document}

\title{Cryptography in the Post-Quantum Era}


\author{Rvail Naveed}

\maketitle

% As a general rule, do not put math, special symbols or citations
% in the abstract or keywords.

\begin{abstract} % ~60 words
In the past decade, there has been extensive development in the quantum computing field. 
Quantum computers are a fundamentally new way of performing calculations and solving problems. 
Due to this, modern cryptography systems are put at risk. This report aims to review the concept of "Post-Quantum Cryptography", 
and what can be done to mitigate the risk to cryptosystems presented by quantum computers.  
\end{abstract}

\begin{IEEEkeywords}
Quantum Computing, Cryptography, Public-Key Encryption
\end{IEEEkeywords}


\section{Introduction}

\IEEEPARstart{S}{ince} the creation of the internet in the 1990's there has been an ever growing need for cryptography. 
It has become a major component of the security and integrity of data. Modern encryption algorithms rely heavily on the hardness of complex 
mathematical problems such as \emph{Integer Factorization} and \emph{Discrete Logarithm Problem}. These problems serve as the basis for widely used cryptosystems such as RSA. 
For classical computers, these problems are very hard problems to solve and require immense amounts of computing power and time to break. 
This is why they are said to be secure by today's standards. However, a phenomenon known as Quantum Computing threatens this security. 
The concept of quantum computing was first put forward by a physicist named Paul Benioff when he proposed a quantum mechanical model of the Turing machine. 
Benioff was the first to show that quantum computing was theoretically possible. Many others built upon his work and it was then that it became known that a 
quantum computer could have the possibility to solve problems in a fundamentally new way that would allow for extreme gains in speed and computing power, 
that would be unparalleled by the classical computers of today.
\\ \\
The rest of this paper aims to review quantum computers, how they operate, what makes current encryption schemes vulnerable,
how quantum computers threaten their security and post-quantum alternatives for cryptosystems.

\section{Quantum Computers}
Unlike classical computers that depend on values being confirmed and definite in order to perform calculations on them, quantum computers can perform calculations 
on variables before they have "settled" into  a state. This allows quantum computers to perform calculations with variables who’s states are undetermined. 
The common example used to illustrate this effect is a spinning coin. A classical computer only knows the value of the coin when it lands, either heads or tails. 
A quantum computer interprets the spinning coin as being both in the state of heads and tails at the same time. 
This phenomenon is called superposition and is one of the main ideas that makes these new computers so powerful. Quantum computers perform operations with \emph{qubits}. 
Qubits are undefined properties of a system and they have the ability to be in a state of superposition, much like the spinning coin. These qubits are then used in special quantum algorithms such as 
\emph{Shor's Algorithm} discussed below, to perform multiple probabilistic calculations simultaneously, cutting down on the time it would take a classical computer to do the same task. \\ \\ 
Another fundamental concept that quantum computers take advantage of is \emph{Quantum Entanglement}. This refers to when 2 or more $(n \geq 2)$ qubits become an n-tuple. When this happens, 
a change in one of the qubits in the tuple will result in a direct change in the other, in some deterministic way. Entanglement also allows for the "daisy-chaining" of multiple entangled qubits 
for exponential computing gain. \\ \\ 
Despite the promising nature of quantum computers, there are some challenges that need to be worked on in order to make it a significant threat to cryptography. 
Quantum computers are much more error prone than their classical counterparts due to \emph{decoherence}. This can happen due to random noise in the environment such as 
heat and vibrations which make the superposition of the qubits decay over time, leading to less reliable functionality. 
However, there is extensive research being done on this subject and it won't be long before a fully functional quantum computer is a reality. 
According to the American National Institute of Standards and Technology (NIST), it is a probable scenario that a quantum computer could be built for a budget of 
one billion dollars by 2030 that could break RSA-2048, a feat that is impossible by classical computers of the current age. 

\section{Current Cryptography}
Current cryptography techniques can be divided into 2 distinct types: \emph{Symmetric} and 
\emph{Asymmetric}. These cryptography techniques rely heavily on the hardness of solving problems such as Integer Factorization and the Discrete 
Logarithm problem.

\subsection{Symmetric}
In symmetric-key cryptography there is only one key that is used to encrypt and decrypt messages between parties. 
The plaintext x is entered into an algorithm called a "cipher" that encrypts the plaintext using the specified key and outputs 
it as "scrambled" ciphertext. This ciphertext is now encrypted and can only be converted back to its original plaintext form by using the 
specified key that it was  encrypted with. Symmetric-key cryptography allows for a respectable amount of security and efficiency. 
However the benefits carry a risk as there are concerns of the keys being intercepted when they are being transmitted/distributed 
between parties. 
\subsection{Asymmetric}
Asymmetric cryptography is commonly referred to as public-key cryptography. This is because it uses a 2-key pair to encrypt and 
decrypt messages. Public-key cryptography uses a \emph{public key} that can be shared over any channel and a \emph{private key} 
that is kept secret by the intended sender/recipient. The sender uses the receivers public key to encrypt a message before transmission, 
and the receiver uses their own private key to decrypt the message and view it in its plaintext form. Public-key algorithms have seen 
wide adoption and serve as a backbone to many services today such as the Secure Socket Layer and HTTPS technology. 
Digital signatures are also commonly verified with public-key techniques. 
\subsection{Integer Factorization Problem}
RSA first proposed by Rivest, Shamir and Adleman in the 1970's, is one of the world's first public key cryptosystems. RSA is often used in combination with symmetric key algorithms for 
maximum security and speed. RSA encryption is designed to be easy to compute in one direction and difficult to compute in the opposite direction. Such functions are often called \emph{trapdoor}
or \emph{one-way} functions. RSA relies on the integer factorization problem via prime numbers. \\ 
The operation of RSA is as follow:
\begin{itemize}
  \item Two large primes are generated $p, q$ such that they are far away from each other, otherwise it would be easier to crack.
  \item The product of the two primes is then found $n=p*q$ and $\phi = (p-1)*(q-1)$
  \item A random number $1<e<\phi$ is selected such that the greatest common divisor (gcd) = $gcd(e, \phi) = 1$
  \item Compute a unique integer $1<d<\phi$ such that $e*d = 1(mod~\phi)$
  \item We then have $(d, n)$ and $(e, n)$ as the private and public key's respectively.
  \item Encrypt message \emph{m} using $c=m^e*mod~n$
  \item Decrypt: $m=c^d mod~n$
\end{itemize}

\subsection{Discrete Logarithm Problem}
Given a finite cycle group $Z^*p$ of order $p-1$ and a primitive element $\alpha \in Z^*p$ and another element $\beta \in Z^*p$, the DLP is determining an integer $1 \leq x \leq p-1$
such that $\alpha^x \equiv \beta mod p$. $x$ is called the discrete logarithm of $\beta$ to the base $\alpha$. 
$$x = log_\alpha \beta mod p$$ The difficuly of solving DLP relies in finding an $x$ that satisfies the equation. DLP can prove to be a very hard problem to solve, if the parameters are large.


\subsection{Security}
Both symmetric and public-key cryprography are considered to be very secure by today's standards. The inherent security of symmetric cryptography is based on the length of the keys used for encryption/decryption. A 128 bit key length would take billions of years
to crack on classical hardware. NIST recommends a move to 256 bit key lengths, which are thought to be theoreticallly resistant to Quantum attacks. The same cannot be said for public-key algorithms such as RSA and Elliptic Curve 
Crptography as quantum algorithms such as \emph{Shor's Algorithm} and \emph{Grover's Algorithm} aim to drastically reduce the amount of time it takes to crack them. With classical hardware, prime factorization is extremely time consuming and
unimaginable, with the best time being $O(exp(\sqrt[n]{\frac{64}{9}n(logn)^2}))$. However, with Shor's algorithm this time is significantly reduced to $O(n^3logn)$, where $n$ the bits of the product $p*q$.

\section{Shor's Algorithm}
In 1994, Peter Shor developed a quantum polynomial-time quantum algorithm for integer factorization. This algorithm 
described a process for finding factors of a number, even a very large one in a very short amount of time. Integer Factorization
serves as the basis for many of the worlds currently most secure crpytography algorithms, such as RSA. In Layman's terms, Shor's algorithm takes
a guess at a number that could share factors with $N$ (but most likely doesn't) and transforms that guess into a much better one
that is very likely to be a number that shares factors with $N$. Understanding how algorithms like RSA operate is fundamental to understanding why Shor's Algorithm works.
Shor's algorithm leverages the concept of quantum superposition mentioned earlier to
compute factors of these very large numbers at a scale that is unmatched by modern classical computers. In order to compute factors of N
\begin{itemize}
  \item Consider $N$, a very large number (eg: An RSA public key)
  \item In order to crack RSA, we neeed to find factors of $N$ such that $f(x)=(x^r) mod N$.
  \item Shor's Algorithm shifts the focus to finding period $r$ i.e: $f(x)=f(x+r)$. 
  \item The algorithm uses something known as a \emph{Quantum Fourier Transform}(QFT).
  \item A quantum computer can be in many states simultaneously, which leads to very fast computing.
  \item This allows the computer to calculate the period $r$ for all points simultaneously.
  \item 
  Then the QFT essentially takes these values, transforms them into waves in such a way that they 
  destructively interfere with each other, leaving only the correct value of $r$.
  \item This $r$ can then directly be used to solve for a factor of $N$, thus breaking the encryption.
\end{itemize}
Shor's algorithm also applies to cryptography schemes that depend on the \emph{Discrete Logarithm Problem}.

\section{Quantum Key Distribution}
Quantum Key distribution or QKD, first proposed in 1970, is a means of secure communication over a channel. This is achieved via a crytographic protocol based on the fundamentals of quantum mechanincs.
QKD enables 2 or more people to share messages using a shared random key, much like traditional cryptography. In QKD , information is encoded in single photons of light which require special hardware to create and transmit.
QKD's use lies in generating secure cryptographic keys for use in conjunction with other cryptograpghy schemes such as RSA, it would be too unreliable and inefficient at directly sending information.
QKD in essence, works as follows; Two users Alice and Bob wish to communicate with each other but are unsure if someone is eavesdropping on their conversation. Alice tries to send a message to Bob using a "bitstream" that is encoded in a base of 
her choosing. If Bob also chooses the same base to decode the message, then their results correlate, otherwise they are random and are discarded. If someone tried to intercept the communication, they would introduce errors into the bitstream, this is discussed further below.
This approach to key distribution comes with a number of advantages, namely that a secret key can be created in a provably secure way to prevent any eavesdropping of private information, eavesdropping is easily detectable and perhaps most importantly, QKD has the ability to 
easily integrate with current cryptography to provide additional security. In this scenario, the interceptor would not only have to break the quantum key but also the classical one. \\ \\
There are two main types of QKD protocols: \emph{Prepare and Measure} and \emph{Entanglement Based}.
\subsection{Prepare and Measure}
Measurement one of the fundamental concepts of quantum mechanics. Measurement of a quantum state changes that state in some way. This fact of quantum mechanics can be taken advantage of to detect eavesdropping,
as the very act of eavesdropping requires some level of measurement. Prepare and Measure techniques also allow us to see \emph{how much} information has been intercepted. The most common prepare and measure protocol is BB84 which is discussed further below.
\subsection{Entanglement Based}
Entanglement based protocols exploit the quantum concept of entanglement which was explained earlier. Two or more qubits states become dependant on each other and they are in a joint state of superposition. Trying to determine the state of one of those qubits will directly result in a change in the other, making it very easy to
detect eavesdropping. Theoretically, entanglement based protocols are  much more secure than their prepare and measure counterparts, however it is currently unfeasible as managing and creating entangled qubits is very difficult.
\subsection{BB84 Protocol}
BB84 was the worlds first Quantum Key Distribution protocol. It was created in 1984 by Bennet and Brassard. BB84's operation can be divided into three layers:
\subsubsection{Physical Layer}
This layer is the most hardware intensive portion of BB84. This is where communication between two parties, Alice and Bob, takes place. Alice chooses random photons to send to Bob, these can either be encoded with a 1 with 50\% probability or a 0 with 50\% probability.
She can then encode the photons in either "base" \emph{X} or \emph{Y}, each with equal probability. These encoded photons are then sent over a quantum channel to Bob. Bob does not know what base Alice used or any information about how the qubits were encoded. He measures each bit on a random basis in either \emph{X} or \emph{Y} each with 50\% probability.
This leads to Bob having 75\% of the correct bases for the qubits.
\subsubsection{Key-Extraction}
In the next layer BB84 becomes classical to etract the key from the qubits that were sent by Alice. This layer consists of 4 sublevels. In the first sublevel called \emph{sifting}, Alice and Bob reveal which bases they used and compare them over a public channel.
They then proceed to discard the bits that do not correlate with each other. \\ \\
In the 2nd sublevel \emph{authentication}, Alice and Bob compare a subset of their bits to determine if their communication channel was compromised. If there is a higher error than can be accounted for by random noise, then the channel is believed to have been intercepted by a third party. This is because in order for an attacker to intercept the qubits, 
they must also guess the base wrong 50\% of the time, leading them to forward some incorrect bits to Bob. \\ \\
\emph{Error correction} algorithms are then applied to reduce the effect of random noise on the data. \\ \\
In the final sublayer \emph{Privacy amplification}, the key generated from BB84 is combined with classical cryptography algorithms to reduce the effects of any undetected, minor eavesdropping and to maximize the security of QKD. This is done by encrypting the key itself using a scheme such as RSA or AES. This will require two levels of decryption. \\ \\
\subsubsection{Key-Application}
The key-application layer is then responsible for using this generated key to encode data.
\subsection{E91 Protocol}
The E91 protocol is similar in nature to BB84 except it operates on entangled qubits. Alice and Bob both possess half of an entangled state. In this scenario there is nothing for an eavesdropper to intercept as the state of a qubit only settles after it has been measuered. As before Alice and Bob choose 1 iut of 2 bases to measure in, each with 50\% probaility of getting chosen.
After the qubits have been measured, they both share what bases they used over a public channel. If the qubits that were measured in different bases violate Bell inequalities, then states of the qubits remain entangled and it can be concluded that to eavesdropping has occured and the channel is secure. This is again, much more secure in theory than BB84 but it is impractical of the aforementioned
decoherence problem of entangled qubits.

\section{Post Quantum Public-Key Algorithms}
Along with the proposed methods of quantum key distribution, there has also been extensive research done into cryptography schemes that would be suitably secure for use in an age where quantum computers become powerful enough to bypass tranditional cryptography. There are three main areas of interest:
\begin{itemize}
  \item Lattice-Based Cryptography
  \item Code-Based Cryptography
  \item Multivariate Polynomial Cryptography
\end{itemize}
These methods do not rely on the integer factorization and discrete logarithm problems used by so many cryptosystems today. Making them theoretically, viable candidates to replace these schemes.

\subsection{Lattice Based}
A lattice can be described as a grid of evenly spaces points stretching out to infinity in all directions.
Lattice based cryptography has seen promising development in recent years. Instead of basing the algorithm on multiplying primes, lattice based solutions focus on multiplying matrices/vectors.
The security proofs are related to hard math problems that involve lattice structures. \\
A \emph{basis} of a lattice is a small subset of vectors from the original lattice that can be used to reproduce the original grid of points.
Basis' are derived to conserve computer memory, as lattices tend to be very storage heavy. The basis of a lattice is \emph{short} when it contains short vectors, likewise when the basis is long. Some lattice problems are described below
as well what a public-key system based on lattice problems would look like. Pursuing a public-key cryptosystem has many advantages, lattice problems are some of the most well understood and researched types of hard math problems, dating back
to the 1800's.
\subsubsection{Short vector problem}
In this problem a long basis is constructed from a lattice L. The task is then to find a point somewhere in the lattice's grid as close as possible to the point of origin.
The hardness of this problem lies in the relative difficulty of finding a short point in a long basis. Furthermore, a lattice can be very large consisting of thousands of points, so finding a combination of basis vectors that concurrently leads to those
thousands of points small turns into an extrenuous task. 
\subsubsection{Short Basis Problem}
\begin{itemize}
  \item Given a long basis for a lattice L.
  \item Find a short basis in L.
\end{itemize}
\subsubsection{Closest Vector Problem}
\begin{itemize}
  \item Given a long basis for a lattice L.
  \item A randomly chosen challenge point $P$ is also selected.
  \item Find the closest point in L to challenge $P$.
\end{itemize}
\subsubsection{Public-Key Scheme}
Encryption is done by selecting a vector $v$ as plaintext. This vector would then be hidden by adding some error vector to it such that $c=v+e$.
Decryption of the ciphertext would then involve solving the shortest vector problem mentioned aboce using $c$ as the input. However this approach does not lead to secure schemes. When worst-to-average reduction schemes
are applied, lattice based solutions also tend to lose their efficency. However NTRU, an open source lattice based cryptosystem seems promising. NTRU is very fast and efficient, but there could be unknown security flaws and no security proof exists for NTRU.
\subsection{Code-Based}
Code-based solutions make use off error correcting codes. They are based on the inherent difficulty of decoding linear codes.
Code bases solutions are thought to be resilient against quantum attacks if their key size is increased by a factor of 4. \\
The McEliece Goppa Code cryptosystem (MECS) is one of the most prominent code based solutions. In MECS encryption and decryption have quadratic complexity, which is on par with RSA. The main downside to MECS is the very large key size compared to other crypto algorithms.
Common key sizes can be as big as 0.5Mb, vs 0.1Mb for RSA and just 0.02Mb for Elliptic Curve Cryptography (ECC). There is also no known polynomial time algorithm for decoding liner block codes and the problem is said to be NP hard, which further increases it's merit as a replacement for
current encryption schemes. However, work needs to be done to improve the security of this system. \\
There are many known attacks against MECS, one of them was proposed by McEliece himself and is based on \emph{information set decoding}. In this attack k co-ordinates are selected from n of the code in a way that no errors affect the selected co-ordinates. This can be done with probability of $(1-\frac{t}{n})^k$.
The plaintext message can then be extracted using simple algebra and the running time of the attack is an unbearable $k^3(1-\frac{t}{n})^k$. This requires MECS to be combined with a semantically secure conversion  to increase its security. Despite these vulnerabilites, MECS still remains a prime candidate for a drop-in replacement for algorithms
such as RSA, should the need arise.
\subsection{Multivariate Polynomial}
A multivariate polynomial is a polynomial with multiple variables such as $-3x^7-v^5y^4+5+6yx^3v^2$. Solving multivariate polynomials are also proven to be NP hard problems. Currently no encryption schemes based on multivariate polynomials exist, however there are some signature schemes implemented based on the concept such as 
Unbalanced Oil and Vinegar(UOV) and Rainbow. The most promising encryption scheme is called SimpleMatrix or ABC. ABC allows for very fast encryption and decryption, however decryption fails with high probabilities and the key size for the scheme is also fairly large. It is also worth noting that many proposed signature schemes have been broken in the past such as the aforementioned UOV.
Multivariate polynomial cryptography is also very young, and it has not been thoroughly tested by cryptanalytic techniques to consider it a feasible option for post-quantum cryptography.

% An example of a floating figure using the graphicx package.
% Note that \label must occur AFTER (or within) \caption. 
% is designed to preserve the operation of \label within \caption
% even when the captionsoff option is in effect. However, because
% of issues like this, it may be the safest practice to put all your
% \label just after \caption rather than within \caption{}.
%
% Reminder: the "draftcls" or "draftclsnofoot", not "draft", class
% option should be used if it is desired that the figures are to be
% displayed while in draft mode.
%
%\begin{figure}[!t]
%\centering
%\includegraphics[width=2.5in]{myfigure}
% where an .eps filename suffix will be assumed under latex, 
% and a .pdf suffix will be assumed for pdflatex; or what has been declared
% via \DeclareGraphicsExtensions.
%\caption{Simulation results for the network.}
%\label{fig_sim}
%\end{figure}

% Note that the IEEE typically puts floats only at the top, even when this
% results in a large percentage of a column being occupied by floats.


% An example of a double column floating figure using two subfigures.
% (The subfig.sty package must be loaded for this to work.)
% The subfigure \label commands are set within each subfloat command,
% and the \label for the overall figure must come after \caption.
% \hfil is used as a separator to get equal spacing.
% Watch out that the combined width of all the subfigures on a 
% line do not exceed the text width or a line break will occur.
%
%\begin{figure*}[!t]
%\centering
%\subfloat[Case I]{\includegraphics[width=2.5in]{box}%
%\label{fig_first_case}}
%\hfil
%\subfloat[Case II]{\includegraphics[width=2.5in]{box}%
%\label{fig_second_case}}
%\caption{Simulation results for the network.}
%\label{fig_sim}
%\end{figure*}
%
% Note that often IEEE papers with subfigures do not employ subfigure
% captions (using the optional argument to \subfloat[]), but instead will
% reference/describe all of them (a), (b), etc., within the main caption.
% Be aware that for subfig.sty to generate the (a), (b), etc., subfigure
% labels, the optional argument to \subfloat must be present. If a
% subcaption is not desired, just leave its contents blank,
% e.g., \subfloat[].


% An example of a floating table. Note that, for IEEE style tables, the
% \caption command should come BEFORE the table and, given that table
% captions serve much like titles, are usually capitalized except for words
% such as a, an, and, as, at, but, by, for, in, nor, of, on, or, the, to
% and up, which are usually not capitalized unless they are the first or
% last word of the caption. Table text will default to \footnotesize as
% the IEEE normally uses this smaller font for tables.
% The \label must come after \caption as always.
%
%\begin{table}[!t]
%% increase table row spacing, adjust to taste
%\renewcommand{\arraystretch}{1.3}
% if using array.sty, it might be a good idea to tweak the value of
% \extrarowheight as needed to properly center the text within the cells
%\caption{An Example of a Table}
%\label{table_example}
%\centering
%% Some packages, such as MDW tools, offer better commands for making tables
%% than the plain LaTeX2e tabular which is used here.
%\begin{tabular}{|c||c|}
%\hline
%One & Two\\
%\hline
%Three & Four\\
%\hline
%\end{tabular}
%\end{table}


% Note that the IEEE does not put floats in the very first column
% - or typically anywhere on the first page for that matter. Also,
% in-text middle ("here") positioning is typically not used, but it
% is allowed and encouraged for Computer Society conferences (but
% not Computer Society journals). Most IEEE journals/conferences use
% top floats exclusively. 
% Note that, LaTeX2e, unlike IEEE journals/conferences, places
% footnotes above bottom floats. This can be corrected via the
% \fnbelowfloat command of the stfloats package.




\section{Conclusion}
Cryptography plays an ever growing role in our society. It is used to secure the transmission and storage of billions of peoples data. Quantum Computers
propose a significant threat to the security of current cryptosystems that we have come to rely on. There are many promising post-quantum alternatives such as the code based
McEliece cryptosystem and Lattice Based cryptography. It is unclear if any of the currently known algorithms will have the capability to serve as drop-in replacements when powerful quantum computers
become a reality. Post-quantum cryptography faces many challenges in order to make it viable and there needs to be extensive research done to improve efficiency and usability. Development requires a collective, community effort to improve these problems
and to build confidence in the security of any schemes proposed. However, the future looks bright as there are breakthroughs being made regularly, and it is improbable that the world will be completely unprepared by the time a sufficently powerful and usable quantum
computer is developed.





% if have a single appendix:
%\appendix[Proof of the Zonklar Equations]
% or
%\appendix  % for no appendix heading
% do not use \section anymore after \appendix, only \section*
% is possibly needed

% use appendices with more than one appendix
% then use \section to start each appendix
% you must declare a \section before using any
% \subsection or using \label (\appendices by itself
% starts a section numbered zero.)
%




% Can use something like this to put references on a page
% by themselves when using endfloat and the captionsoff option.
\ifCLASSOPTIONcaptionsoff
  \newpage
\fi



% trigger a \newpage just before the given reference
% number - used to balance the columns on the last page
% adjust value as needed - may need to be readjusted if
% the document is modified later
%\IEEEtriggeratref{8}
% The "triggered" command can be changed if desired:
%\IEEEtriggercmd{\enlargethispage{-5in}}

% references section

% can use a bibliography generated by BibTeX as a .bbl file
% BibTeX documentation can be easily obtained at:
% http://mirror.ctan.org/biblio/bibtex/contrib/doc/
% The IEEEtran BibTeX style support page is at:
% http://www.michaelshell.org/tex/ieeetran/bibtex/
%\bibliographystyle{IEEEtran}
% argument is your BibTeX string definitions and bibliography database(s)
%\bibliography{IEEEabrv,../bib/paper}
%
% <OR> manually copy in the resultant .bbl file
% set second argument of \begin to the number of references
% (used to reserve space for the reference number labels box)
\begin{thebibliography}{1}

% \bibitem{}
% H.~Kopka and P.~W. Daly, \emph{A Guide to \LaTeX}, 3rd~ed.\hskip 1em plus
%   0.5em minus 0.4em\relax Harlow, England: Addison-Wesley, 1999.

\bibitem{}
Bernstein D.J. (2009) \emph{Introduction to post-quantum cryptography.}
  In: Bernstein D.J., Buchmann J., Dahmen E. (eds) Post-Quantum Cryptography. Springer, Berlin, Heidelberg DOI: 10.1007/978-3-540-88702-7\_1


\bibitem{}
Lidong Chen and Stephen P. Jordan and Yi-Kai Liu and Dustin Moody and Ren{\'e} Peralta and Ray A. Perlner and Daniel Smith-Tone (2016)
  \emph{Report on Post-Quantum Cryptography} DOI: 10.6028/NIST.IR.8105

\bibitem{}
Johannes A. Buchmann, Denis Butin, \emph{The New CodeBreakers: Post-Quantum Cryptography State of the Art} pp 88-104 DOI: 10.1007/978-3-662-49301-4

\bibitem{}
O. Grote, A. Ahrens and C. Benavente-Peces, \emph{"A Review of Post-quantum Cryptography and Crypto-agility Strategies,"} 2019 International Interdisciplinary PhD Workshop (IIPhDW), 
Wismar, Germany, 2019, pp. 115-120.

\bibitem{}
Daniel Moskovich, \emph{An Overview of the State of the Art for Practical Quantum Key Distribution} Available: arXiv:1504.05471

\bibitem{}
Marek Repka — Pavol Zajac, \emph{Overview of the McEliece Cryptosystem and its Security} DOI: 10.2478/tmmp-2014-0025

\bibitem{}
Erdem Alkim and L{\'e}o Ducas and Thomas P{\"o}ppelmann and Peter Schwabe, \emph{Post-quantum Key Exchange - A New Hope}(2015), IACR Cryptology ePrint Archive

\bibitem{}
Dong Pyo Chi, Jeong Woon Choi, Jeong San Kim4and Taewan Kim, \emph{Lattice Based Cryptography for Beginners}, Available: \url{https://eprint.iacr.org/2015/938.pdf}

\bibitem{}
eik Guan Tan, and Jianying Zhou, Senior Member, IEEE, \emph{A Survey of Digital Signingin the Post Quantum Era}, Available: \url{https://eprint.iacr.org/2019/1374.pdf}

\end{thebibliography}


\end{document}


