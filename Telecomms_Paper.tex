\documentclass[journal]{IEEEtran}
% *** CITATION PACKAGES ***
%
\usepackage{cite}
% cite.sty was written by Donald Arseneau
% V1.6 and later of IEEEtran pre-defines the format of the cite.sty package
% \cite{} output to follow that of the IEEE. Loading the cite package will
% result in citation numbers being automatically sorted and properly
% "compressed/ranged". e.g., [1], [9], [2], [7], [5], [6] without using
% cite.sty will become [1], [2], [5]--[7], [9] using cite.sty. cite.sty's
% \cite will automatically add leading space, if needed. Use cite.sty's
% noadjust option (cite.sty V3.8 and later) if you want to turn this off
% such as if a citation ever needs to be enclosed in parenthesis.
% cite.sty is already installed on most LaTeX systems. Be sure and use
% version 5.0 (2009-03-20) and later if using hyperref.sty.
% The latest version can be obtained at:
% http://www.ctan.org/pkg/cite
% The documentation is contained in the cite.sty file itself.






% *** GRAPHICS RELATED PACKAGES ***
%
\ifCLASSINFOpdf
  % \usepackage[pdftex]{graphicx}
  % declare the path(s) where your graphic files are
  % \graphicspath{{../pdf/}{../jpeg/}}
  % and their extensions so you won't have to specify these with
  % every instance of \includegraphics
  % \DeclareGraphicsExtensions{.pdf,.jpeg,.png}
\else
  % or other class option (dvipsone, dvipdf, if not using dvips). graphicx
  % will default to the driver specified in the system graphics.cfg if no
  % driver is specified.
  % \usepackage[dvips]{graphicx}
  % declare the path(s) where your graphic files are
  % \graphicspath{{../eps/}}
  % and their extensions so you won't have to specify these with
  % every instance of \includegraphics
  % \DeclareGraphicsExtensions{.eps}
\fi
% graphicx was written by David Carlisle and Sebastian Rahtz. It is
% required if you want graphics, photos, etc. graphicx.sty is already
% installed on most LaTeX systems. The latest version and documentation
% can be obtained at: 
% http://www.ctan.org/pkg/graphicx
% Another good source of documentation is "Using Imported Graphics in
% LaTeX2e" by Keith Reckdahl which can be found at:
% http://www.ctan.org/pkg/epslatex
%
% latex, and pdflatex in dvi mode, support graphics in encapsulated
% postscript (.eps) format. pdflatex in pdf mode supports graphics
% in .pdf, .jpeg, .png and .mps (metapost) formats. Users should ensure
% that all non-photo figures use a vector format (.eps, .pdf, .mps) and
% not a bitmapped formats (.jpeg, .png). The IEEE frowns on bitmapped formats
% which can result in "jaggedy"/blurry rendering of lines and letters as
% well as large increases in file sizes.
%
% You can find documentation about the pdfTeX application at:
% http://www.tug.org/applications/pdftex





% *** MATH PACKAGES ***
%
%\usepackage{amsmath}
% A popular package from the American Mathematical Society that provides
% many useful and powerful commands for dealing with mathematics.
%
% Note that the amsmath package sets \interdisplaylinepenalty to 10000
% thus preventing page breaks from occurring within multiline equations. Use:
%\interdisplaylinepenalty=2500
% after loading amsmath to restore such page breaks as IEEEtran.cls normally
% does. amsmath.sty is already installed on most LaTeX systems. The latest
% version and documentation can be obtained at:
% http://www.ctan.org/pkg/amsmath





% *** SPECIALIZED LIST PACKAGES ***
%
%\usepackage{algorithmic}
% algorithmic.sty was written by Peter Williams and Rogerio Brito.
% This package provides an algorithmic environment fo describing algorithms.
% You can use the algorithmic environment in-text or within a figure
% environment to provide for a floating algorithm. Do NOT use the algorithm
% floating environment provided by algorithm.sty (by the same authors) or
% algorithm2e.sty (by Christophe Fiorio) as the IEEE does not use dedicated
% algorithm float types and packages that provide these will not provide
% correct IEEE style captions. The latest version and documentation of
% algorithmic.sty can be obtained at:
% http://www.ctan.org/pkg/algorithms
% Also of interest may be the (relatively newer and more customizable)
% algorithmicx.sty package by Szasz Janos:
% http://www.ctan.org/pkg/algorithmicx




% *** ALIGNMENT PACKAGES ***
%
%\usepackage{array}
% Frank Mittelbach's and David Carlisle's array.sty patches and improves
% the standard LaTeX2e array and tabular environments to provide better
% appearance and additional user controls. As the default LaTeX2e table
% generation code is lacking to the point of almost being broken with
% respect to the quality of the end results, all users are strongly
% advised to use an enhanced (at the very least that provided by array.sty)
% set of table tools. array.sty is already installed on most systems. The
% latest version and documentation can be obtained at:
% http://www.ctan.org/pkg/array


% IEEEtran contains the IEEEeqnarray family of commands that can be used to
% generate multiline equations as well as matrices, tables, etc., of high
% quality.




% *** SUBFIGURE PACKAGES ***
%\ifCLASSOPTIONcompsoc
%  \usepackage[caption=false,font=normalsize,labelfont=sf,textfont=sf]{subfig}
%\else
%  \usepackage[caption=false,font=footnotesize]{subfig}
%\fi
% subfig.sty, written by Steven Douglas Cochran, is the modern replacement
% for subfigure.sty, the latter of which is no longer maintained and is
% incompatible with some LaTeX packages including fixltx2e. However,
% subfig.sty requires and automatically loads Axel Sommerfeldt's caption.sty
% which will override IEEEtran.cls' handling of captions and this will result
% in non-IEEE style figure/table captions. To prevent this problem, be sure
% and invoke subfig.sty's "caption=false" package option (available since
% subfig.sty version 1.3, 2005/06/28) as this is will preserve IEEEtran.cls
% handling of captions.
% Note that the Computer Society format requires a larger sans serif font
% than the serif footnote size font used in traditional IEEE formatting
% and thus the need to invoke different subfig.sty package options depending
% on whether compsoc mode has been enabled.
%
% The latest version and documentation of subfig.sty can be obtained at:
% http://www.ctan.org/pkg/subfig




% *** FLOAT PACKAGES ***
%
%\usepackage{fixltx2e}
% fixltx2e, the successor to the earlier fix2col.sty, was written by
% Frank Mittelbach and David Carlisle. This package corrects a few problems
% in the LaTeX2e kernel, the most notable of which is that in current
% LaTeX2e releases, the ordering of single and double column floats is not
% guaranteed to be preserved. Thus, an unpatched LaTeX2e can allow a
% single column figure to be placed prior to an earlier double column
% figure.
% Be aware that LaTeX2e kernels dated 2015 and later have fixltx2e.sty's
% corrections already built into the system in which case a warning will
% be issued if an attempt is made to load fixltx2e.sty as it is no longer
% needed.
% The latest version and documentation can be found at:
% http://www.ctan.org/pkg/fixltx2e


%\usepackage{stfloats}
% stfloats.sty was written by Sigitas Tolusis. This package gives LaTeX2e
% the ability to do double column floats at the bottom of the page as well
% as the top. (e.g., "\begin{figure*}[!b]" is not normally possible in
% LaTeX2e). It also provides a command:
%\fnbelowfloat
% to enable the placement of footnotes below bottom floats (the standard
% LaTeX2e kernel puts them above bottom floats). This is an invasive package
% which rewrites many portions of the LaTeX2e float routines. It may not work
% with other packages that modify the LaTeX2e float routines. The latest
% version and documentation can be obtained at:
% http://www.ctan.org/pkg/stfloats
% Do not use the stfloats baselinefloat ability as the IEEE does not allow
% \baselineskip to stretch. Authors submitting work to the IEEE should note
% that the IEEE rarely uses double column equations and that authors should try
% to avoid such use. Do not be tempted to use the cuted.sty or midfloat.sty
% packages (also by Sigitas Tolusis) as the IEEE does not format its papers in
% such ways.
% Do not attempt to use stfloats with fixltx2e as they are incompatible.
% Instead, use Morten Hogholm'a dblfloatfix which combines the features
% of both fixltx2e and stfloats:
%
% \usepackage{dblfloatfix}
% The latest version can be found at:
% http://www.ctan.org/pkg/dblfloatfix




%\ifCLASSOPTIONcaptionsoff
%  \usepackage[nomarkers]{endfloat}
% \let\MYoriglatexcaption\caption
% \renewcommand{\caption}[2][\relax]{\MYoriglatexcaption[#2]{#2}}
%\fi
% endfloat.sty was written by James Darrell McCauley, Jeff Goldberg and 
% Axel Sommerfeldt. This package may be useful when used in conjunction with 
% IEEEtran.cls'  captionsoff option. Some IEEE journals/societies require that
% submissions have lists of figures/tables at the end of the paper and that
% figures/tables without any captions are placed on a page by themselves at
% the end of the document. If needed, the draftcls IEEEtran class option or
% \CLASSINPUTbaselinestretch interface can be used to increase the line
% spacing as well. Be sure and use the nomarkers option of endfloat to
% prevent endfloat from "marking" where the figures would have been placed
% in the text. The two hack lines of code above are a slight modification of
% that suggested by in the endfloat docs (section 8.4.1) to ensure that
% the full captions always appear in the list of figures/tables - even if
% the user used the short optional argument of \caption[]{}.
% IEEE papers do not typically make use of \caption[]'s optional argument,
% so this should not be an issue. A similar trick can be used to disable
% captions of packages such as subfig.sty that lack options to turn off
% the subcaptions:
% For subfig.sty:
% \let\MYorigsubfloat\subfloat
% \renewcommand{\subfloat}[2][\relax]{\MYorigsubfloat[]{#2}}
% However, the above trick will not work if both optional arguments of
% the \subfloat command are used. Furthermore, there needs to be a
% description of each subfigure *somewhere* and endfloat does not add
% subfigure captions to its list of figures. Thus, the best approach is to
% avoid the use of subfigure captions (many IEEE journals avoid them anyway)
% and instead reference/explain all the subfigures within the main caption.
% The latest version of endfloat.sty and its documentation can obtained at:
% http://www.ctan.org/pkg/endfloat
%
% The IEEEtran \ifCLASSOPTIONcaptionsoff conditional can also be used
% later in the document, say, to conditionally put the References on a 
% page by themselves.




% *** PDF, URL AND HYPERLINK PACKAGES ***
%
%\usepackage{url}
% url.sty was written by Donald Arseneau. It provides better support for
% handling and breaking URLs. url.sty is already installed on most LaTeX
% systems. The latest version and documentation can be obtained at:
% http://www.ctan.org/pkg/url
% Basically, \url{my_url_here}.




% *** Do not adjust lengths that control margins, column widths, etc. ***
% *** Do not use packages that alter fonts (such as pslatex).         ***
% There should be no need to do such things with IEEEtran.cls V1.6 and later.
% (Unless specifically asked to do so by the journal or conference you plan
% to submit to, of course. )


% correct bad hyphenation here
%\hyphenation{op-tical net-works semi-conduc-tor}


\begin{document}

\title{Cryptography in the Post-Quantum Era}


\author{Rvail Naveed}

\maketitle

% As a general rule, do not put math, special symbols or citations
% in the abstract or keywords.

\begin{abstract} % ~60 words
In recent years there has been a surge of interest and development in quantum computing.
Quantum computers have the ability to solve complex mathematical problems that classical computers are not able to.
This puts modern encryption systems at risk. This report aims to review the concept of "post-quantum cryptography" and what it would look like in the age of
large, scalable quantum computers.
\end{abstract}

\begin{IEEEkeywords}
Quantum Computing, Cryptography, Public-Key Encryption
\end{IEEEkeywords}


\section{Introduction}

\IEEEPARstart{S}{ince} the rise of the internet in the 1990's, there has been a very urgent need
for crpytography. Cryptography, derived from the Greek word for "hidden" and "writing" has become a staple
of every modern technology in use today. It is essential to countless businesses and the security, integrity 
and confidentiality of our ever-growing online presence as a society. Over the past few decades there have been 
significant strides made in the cryptography world (such as RSA and AES). Modern encryption algorithms rely on the
hardness of solving one of two problems: \emph{Integer Factorization} and the \emph{Discrete Logarithm Problem}.
These problems prove to be very difficult to solve modern classical computers as the power and time to solve them is immense.
However, a relatively new phenomenon, Quantum-Computers, threaten this security. \\ \\
Quantum computing began in the early 1980's when a physicist named Paul Benioff proposed a quantum mechanical model
of the Turing machine. Richard Feynman, a theoretical physicist later brought up the idea that a quantum
computer may be able to perform tasks that a classical computer did not have the capacity for. 
\\ \\
This paper aims to review Quantum Computers, how they operate, modern encryption algorithms and how a quantum computer
threatens their security. 

\section{Quantum Computers}
Quantum Computers can perform calculations based on the probabiltity an objects state before it's measured.
Classical computers encode data in bits. These bits can have \emph{either} a state of $0$ or $1$. Quantum computers, on the other hand
use \emph{qubits} to perform their caclulations. Qubits are undefined properties of a system and have the ability to be in superposition.
This means that a qubit can be in both states at the same time. Superposition can be thought of as a spinning coin, as the coin is spinning, 
it is neither a heads or a tails until it lands. Qubits are plugged into special algorithms that allows them to crunch through a vast number of outcomes
simultaneously. To put this into perspective, 1 qubit can be in a superposition of 2 states, 3 qubits can be in a superposition of 8 states and 
50 qubits can be in a superposition of 250 = 1,125,899,906,842,624 states at once. As of 2019 the most powerful quantum computer is owned by IBM with 53 qubits. \\ \\
Another phenomenon that makes quantum computing so powerful is \emph{Quantum Entanglement}. Entanglement allows for one or more qubits to become a pair such that
a change in one qubit will directly result in a change in the other. THis happens, regardless of the distance netween them. Quantum Computers can harness the power of
entangled qubits in a "daisy chain" leading to exponential increases in computing power and speed. \\ \\
However Quantum Computers are much more error prone than their classical counterparts due to \emph{decoherence}.
Qubits are very susceptible to errors from outside factors such 
as heat and vibrations. This causes their quantum behvaiour to decay and become unreliable. Qubits are very fragile and any noise can cause problems in their superposition.
Despite best efforts, researchers have found it difficult to reduce errors due to noise. \\ \\
There is extensive research being done on Quantum Computers and it won't be long before quantum computers are at a point where they become useful. It is estimated that by 2030 a Quantum
computer could be built for a budget of a billion dollars that could crack RSA-2048.

\section{Current Cryptography}
Current cryptography techniques can be divided into 2 distinct types: \emph{Symmetric} and 
\emph{Asymmetric}. These cryptography techniques rely heavily on the hardness of solving problems such as Integer Factorization and the Discrete 
Logarithm problem.

\subsection{Symmetric}
%Grover's Algorithm
Symmetric-key cryptography algorithms rely on a single key to encrypt and decrypt messages. This key can be shared between two or more users.
The plaintext message is entered into a "cipher" that encrypts the message using the key and outputs it as ciphertext. In much the same way,
the ciphertext is decrypted using the key and can be converted back to it's plaintext form. \\ \\
Symmetric Cryptography allows for fairly high security with fast speeds,
however there is a significant risk of keys being intercepted while they are being dsitributed.
\subsection{Asymmetric}
%Shor's Algorithm
Also known as public-key cryptography, it makes use of two keys compared to the one in symmetric. The pair of keys consists of a \emph{public key} and a \emph{private key}.
The sender uses the receivers public key to encrypt a message before sending it. The receiver then decrypts the message using their own related private key. Public-key cryptography is essential to
internet applications such as the Secure Socket Layer, Transport Layer Security and HTTPS. \\ \\ 
Digital signatures can also be verified using public-key techniques. Digital signatures are a way of validating the integrity of data.
A hash is of the data to be signed is generated, this hash is then encrypted using the private key. The hash combined with other metadata is then what's known as the digital signature.
When the recipient recieves the message, it can be verified by running the message through the same hashing algorithm. A match indicated a valid, untampered message.
\subsection{Integer Factorization Problem}

\subsection{Discrete Logarithm Problem}

\subsection{Current Security}
Both symmetric and public-key cryprography are considered to be very secure by today's standards. However this security is threatened as we move towards a reality where Quantum Computers 
are more powerful. The inherent security of symmetric cryptography is based on the length of the keys used for encryption/decryption. A 128 bit key length would take billions of years
to crack on classical hardware. NIST recommends a move to 256 bit key lengths, which are thought to be theoreticallly resistant to Quantum attacks. The same cannot be said for public-key algorithms such as RSA and Elliptic Curve 
Crptography as quantum algorithms such as \emph{Shor's Algorithm} and \emph{Grover's Algorithm} aim to drastically reduce the amount of time it takes to crack them.

\section{Shor's Algorithm - This defo needs to be better}
In 1994, Peter Shor developed a quantum polynomial-time quantum algorithm for integer factorization. This algorithm 
described a process for finding factors of a number, even a very large one in a very short amount of time. Integer Factorization
serves as the basis for many of the worlds currently most secure crpytography algorithms, such as RSA. In Layman's terms, Shor's algorithm takes
a guess at a number that could share factors with $N$ (but most likely doesn't) and transforms that guess into a much better one
that is very likely to be a number that shares factors with $N$. Shor's algorithm leverages the concept of quantum superposition mentioned earlier to
compute factors of these very large numbers at a scale that is unmatched by modern classical computers. In order to compute factors of N
\begin{itemize}
  \item Consider $N$, a very large number (eg: An RSA public key)
  \item In order to crack RSA, we neeed to find factors of $N$ such that $f(x)=(x^r) mod N$.
  \item Shor's Algorithm shifts the focus to finding period $r$ i.e: $f(x)=f(x+r)$. 
  \item The algorithm uses something known as a \emph{Quantum Fourier Transform}(QFT).
  \item A quantum computer can be in many states simultaneously, which leads to very fast computing.
  \item This allows the computer to calculate the period $r$ for all points simultaneously.
  \item 
  Then the QFT essentially takes these values, transforms them into waves in such a way that they 
  destructively interfere with each other, leaving only the correct value of $r$.
  \item This $r$ can then directly be used to solve for a factor of $N$, thus breaking the encryption.
\end{itemize}
Shor's algorithm also applies to cryptography schemes that depend on the \emph{Discrete Logarithm Problem}.

\section{Grover's Algorithm}
What it works on, how it operates

\section{Algorithms Threatened by the Quantum Era}

\section{Post Quantum Crpytography}

% An example of a floating figure using the graphicx package.
% Note that \label must occur AFTER (or within) \caption.
% For figures, \caption should occur after the \includegraphics.
% Note that IEEEtran v1.7 and later has special internal code that
% is designed to preserve the operation of \label within \caption
% even when the captionsoff option is in effect. However, because
% of issues like this, it may be the safest practice to put all your
% \label just after \caption rather than within \caption{}.
%
% Reminder: the "draftcls" or "draftclsnofoot", not "draft", class
% option should be used if it is desired that the figures are to be
% displayed while in draft mode.
%
%\begin{figure}[!t]
%\centering
%\includegraphics[width=2.5in]{myfigure}
% where an .eps filename suffix will be assumed under latex, 
% and a .pdf suffix will be assumed for pdflatex; or what has been declared
% via \DeclareGraphicsExtensions.
%\caption{Simulation results for the network.}
%\label{fig_sim}
%\end{figure}

% Note that the IEEE typically puts floats only at the top, even when this
% results in a large percentage of a column being occupied by floats.


% An example of a double column floating figure using two subfigures.
% (The subfig.sty package must be loaded for this to work.)
% The subfigure \label commands are set within each subfloat command,
% and the \label for the overall figure must come after \caption.
% \hfil is used as a separator to get equal spacing.
% Watch out that the combined width of all the subfigures on a 
% line do not exceed the text width or a line break will occur.
%
%\begin{figure*}[!t]
%\centering
%\subfloat[Case I]{\includegraphics[width=2.5in]{box}%
%\label{fig_first_case}}
%\hfil
%\subfloat[Case II]{\includegraphics[width=2.5in]{box}%
%\label{fig_second_case}}
%\caption{Simulation results for the network.}
%\label{fig_sim}
%\end{figure*}
%
% Note that often IEEE papers with subfigures do not employ subfigure
% captions (using the optional argument to \subfloat[]), but instead will
% reference/describe all of them (a), (b), etc., within the main caption.
% Be aware that for subfig.sty to generate the (a), (b), etc., subfigure
% labels, the optional argument to \subfloat must be present. If a
% subcaption is not desired, just leave its contents blank,
% e.g., \subfloat[].


% An example of a floating table. Note that, for IEEE style tables, the
% \caption command should come BEFORE the table and, given that table
% captions serve much like titles, are usually capitalized except for words
% such as a, an, and, as, at, but, by, for, in, nor, of, on, or, the, to
% and up, which are usually not capitalized unless they are the first or
% last word of the caption. Table text will default to \footnotesize as
% the IEEE normally uses this smaller font for tables.
% The \label must come after \caption as always.
%
%\begin{table}[!t]
%% increase table row spacing, adjust to taste
%\renewcommand{\arraystretch}{1.3}
% if using array.sty, it might be a good idea to tweak the value of
% \extrarowheight as needed to properly center the text within the cells
%\caption{An Example of a Table}
%\label{table_example}
%\centering
%% Some packages, such as MDW tools, offer better commands for making tables
%% than the plain LaTeX2e tabular which is used here.
%\begin{tabular}{|c||c|}
%\hline
%One & Two\\
%\hline
%Three & Four\\
%\hline
%\end{tabular}
%\end{table}


% Note that the IEEE does not put floats in the very first column
% - or typically anywhere on the first page for that matter. Also,
% in-text middle ("here") positioning is typically not used, but it
% is allowed and encouraged for Computer Society conferences (but
% not Computer Society journals). Most IEEE journals/conferences use
% top floats exclusively. 
% Note that, LaTeX2e, unlike IEEE journals/conferences, places
% footnotes above bottom floats. This can be corrected via the
% \fnbelowfloat command of the stfloats package.




\section{Conclusion}
The conclusion goes here.





% if have a single appendix:
%\appendix[Proof of the Zonklar Equations]
% or
%\appendix  % for no appendix heading
% do not use \section anymore after \appendix, only \section*
% is possibly needed

% use appendices with more than one appendix
% then use \section to start each appendix
% you must declare a \section before using any
% \subsection or using \label (\appendices by itself
% starts a section numbered zero.)
%


\appendices
\section{Proof of the First Zonklar Equation}
Appendix one text goes here.

% you can choose not to have a title for an appendix
% if you want by leaving the argument blank
\section{}
Appendix two text goes here.


% use section* for acknowledgment
\section*{Acknowledgment}


The authors would like to thank...


% Can use something like this to put references on a page
% by themselves when using endfloat and the captionsoff option.
\ifCLASSOPTIONcaptionsoff
  \newpage
\fi



% trigger a \newpage just before the given reference
% number - used to balance the columns on the last page
% adjust value as needed - may need to be readjusted if
% the document is modified later
%\IEEEtriggeratref{8}
% The "triggered" command can be changed if desired:
%\IEEEtriggercmd{\enlargethispage{-5in}}

% references section

% can use a bibliography generated by BibTeX as a .bbl file
% BibTeX documentation can be easily obtained at:
% http://mirror.ctan.org/biblio/bibtex/contrib/doc/
% The IEEEtran BibTeX style support page is at:
% http://www.michaelshell.org/tex/ieeetran/bibtex/
%\bibliographystyle{IEEEtran}
% argument is your BibTeX string definitions and bibliography database(s)
%\bibliography{IEEEabrv,../bib/paper}
%
% <OR> manually copy in the resultant .bbl file
% set second argument of \begin to the number of references
% (used to reserve space for the reference number labels box)
\begin{thebibliography}{1}

\bibitem{IEEEhowto:kopka}
H.~Kopka and P.~W. Daly, \emph{A Guide to \LaTeX}, 3rd~ed.\hskip 1em plus
  0.5em minus 0.4em\relax Harlow, England: Addison-Wesley, 1999.

\end{thebibliography}

\end{document}


